\documentclass[10pt]{beamer}

\usetheme{metropolis}
\usepackage{appendixnumberbeamer}

\usepackage{booktabs}
\usepackage[scale=2]{ccicons}

\usepackage{pgfplots}
\usepgfplotslibrary{dateplot}

\usepackage{xspace}
\newcommand{\themename}{\textbf{\textsc{metropolis}}\xspace}

\title{Optimización y profiling de aplicaciones HPC}
\subtitle{Usando herramientas libres}
\date{\today}
\author{Emilio J. Padrón González}
 \institute{\href{mailto:emilioj@udc.gal}{\nolinkurl{emilioj@udc.gal}}
   -- \url{http://gac.udc.es/~emilioj}\\Grupo de Arquitectura de
   Computadores -- Universidade da Coruña}

\begin{document}

\maketitle

\begin{frame}{Contenidos}
  \setbeamertemplate{section in toc}[sections numbered]
  \tableofcontents[hideallsubsections]
\end{frame}

\section{Introducción}

\frame{
}

\section{Optimización y profiling de aplicaciones secuenciales}

\frame{
  \frametitle{\insertsectionhead}

  \begin{itemize}
  \item Valgrind
  \item Gproff
  \end{itemize}
}

\subsection{Evaluación del tiempo de ejecución y deteccion de «hot spots»}

\frame{
}

\subsection{Evaluación del uso de memoria}

\frame{
}

\subsection{Estudio del rendimiento de la memoria caché}

\frame{
}

\subsection{Estrategias para la detección de oportunidades de
  vectorización y paralelización}

\frame{
}

\section{Optimización y profiling de aplicaciones paralelas}

\frame{
  \frametitle{\insertsectionhead}

  \begin{itemize}
  \item Paraver, Extrae y Dimemás
  \item ompP
  \end{itemize}
}

\subsection{Memoria compartida (OpenMP)}

\frame{
}

\subsubsection{Optimización del balanceo de la carga}

\frame{
}

\subsubsection{Estudio y optimización de las comunicaciones}

\frame{
}

\subsection{Memoria distribuida (MPI)}

\frame{
}

\subsubsection{Optimización del balanceo de la carga}

\frame{
}

\subsubsection{Estudio y optimización de las comunicaciones}

\frame{
}

\subsection{Escenario híbrido: mem. compartida + mem. distribuida}

\frame{
}

\subsubsection{Optimización del balanceo de la carga}

\frame{
}

\subsubsection{Estudio y optimización de las comunicaciones}

\frame{
}

\end{document}
